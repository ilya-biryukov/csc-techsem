\documentclass{article}
\usepackage[utf8]{inputenc}
\usepackage[english,russian]{babel}
\usepackage{amsmath,amsthm,amssymb}
\usepackage{tikz}
\usepackage{fullpage}

\newtheorem{theorem}{Теорема}
\begin{document}
  \begin{theorem}[Теорема Пифагора]
    Сумма квадратов катетов прямоугольного треугольника равна квадрату гипотенузы.
    \begin{equation*}
      a^2 + b^2 = c^2
    \end{equation*}
  \end{theorem}
  \begin{proof}
    Рассмотрим прямоугольный треугольник $ABC$. Обозначим за $a$, $b$ длины катетов, за
    $c$ длину гипотенузы.
    \begin{center}
      \begin{tikzpicture}
        \coordinate (A) at (0,0);
        \coordinate (B) at (0,3);
        \coordinate (C) at (4,0);

        \draw (A)
          -- (B) node[midway, left=1]{$a$}
          -- (C) node[midway, right=1, above=1]{$c$}
          -- (A) node[midway, below=1]{$b$};
        \draw (A) node[below=1, left=1] {$A$};
        \draw (B) node[above=1, left=1] {$B$};
        \draw (C) node[below=1, right=1] {$C$};
      \end{tikzpicture}
    \end{center}

    Проведём перпендикуляр из точки $A$ на гипотенузу $BC$. Обозначим точку пересечения
    перпендикуляра и $BC$ за $D$, длину отрезка $BD$ за $x$, длину отрезка $DC$ за $y$.
    \begin{center}
      \begin{tikzpicture}
        \coordinate (A) at (0,0);
        \coordinate (B) at (0,3);
        \coordinate (C) at (4,0);
        \coordinate (D) at (1.44,1.92);

        \draw (A)
          -- (B) node[midway, left=1]{$a$}
          -- (D) node[midway, right=1]{$x$}
          -- (C) node[midway, right=1, above=1]{$y$}
          -- (A) node[midway, below=1]{$b$};
        \draw[fill=blue!30] (A)
          -- (0.5,0) arc (0:53.13:0.5)
          -- cycle;
        \draw[fill=green!30] (A)
          -- (0,0.5) arc (90:53.13:0.5)
          -- cycle;
        \draw[fill=green!30] (C)
          -- (3.5, 0) arc (180:143.13:0.5)
          -- cycle;
        \draw[fill=blue!30] (B)
          -- (0, 2.5) arc (270:323.13:0.5)
          -- cycle;
        \draw[color=red] (A) -- (D);
        \draw (A) node[below=1, left=1] {$A$};
        \draw (B) node[above=1, left=1] {$B$};
        \draw (C) node[below=1, right=1] {$C$};
        \draw (D) node[above=1, right=1] {$D$};
      \end{tikzpicture}
    \end{center}

    Заметим, что треугольники $ABC$ и $ABD$ подобны(по трём углам). Из этого подобия
    получаем:
    \begin{equation*}
      \frac{x}{a} = \frac{a}{c} \implies x = \frac{a^2}{c}
    \end{equation*}

    Аналогично, из подобия треугольников $ABC$ и $ADC$ получаем:
    \begin{equation*}
      \frac{y}{b} = \frac{b}{c} \implies y = \frac{b^2}{c}
    \end{equation*}

    Пользуясь тем, что $c$~--- это сумма $x$ и $y$, получаем:
    \begin{align*}
      c &= x + y \\
      c &= \frac{a^2 + b^2}{c} \\
      c^2 &= a^2 + b^2
    \end{align*}

  \end{proof}
\end{document}
